% Curriculum de Ejemplo
\documentclass[11pt, a4paper]{moderncv}
%Estilo del paquete moderncv, otra opcion es classic
\moderncvstyle{casual}
\moderncvcolor{blue}
%Codificacion
\usepackage[utf8]{inputenc}
%\usepackage{hyperref}
%Ajuste de margenes de pagina
\usepackage[scale=0.8]{geometry}
\AtBeginDocument{\recomputelengths}

\firstname{Nelson Efrain A.}
\familyname{Cruz}
\title{Currículum Vit\'ae}
\address{Villa Lavalle - Rio Cachi 310}{CP 4400 Salta Capital}{Argentina}
\mobile{0387-154899606}
\social[github]{nerones}
\social[linkedin]{nelsonefraincruz}
%\photo[64pt]{picture_file}
%---------------------
% Contenido
%---------------------
\begin{document}

\maketitle

\section{Datos Personales}

\cvlistitem{\textbf{Nombre:} Nelson Efrain Abraham}
\cvlistitem{\textbf{Apellido:} Cruz}
\cvlistitem{\textbf{DNI:} 31853992}
\cvlistitem{\textbf{Fecha de Nacimiento:} 13 de Noviembre de 1985}
\cvlistitem{\textbf{Dirección:} Villa Lavalle - Rio Cachi 310, CP 4400, Salta Capital}
\cvlistitem{\textbf{Teléfonos:} 154899606}
\cvlistitem{\textbf{Correo:} neac03@gmail.com}
%\cvlistitem{\textbf{Web:}  www.ejemplo.net}

\section{Formación académica}

%En los cventry debe haber 6 llaves, aunque no tengamos mas que introducir
%y esten vacias
\cventry{1999--2004}{Técnico Electrónico} {Escuela de Educaci\'on T\'ecnica Nº 5138 Alberto Einstein}{Salta Capital}{}{Aparte de conocimientos en electrónica también fui formado de manera básica en instalaciones eléctricas, carpintería y herrería.}
\cventry{2005--2013}{Computador Universitario}{Universidad Nacional de Salta}{Salta}{}{Titulo orientado a la programación en general con fuerte trasfondo de conocimientos matemáticos, se puede ver el trabajo final de la carrera en \href{https://github.com/nerones/JDBGM}{github.com/nerones/JDBGM}}
\cventry{2013--2016}{Licenciatura en Análisis Sistemas}{Universidad Nacional de Salta}{Salta}{}{}
%\section{Otros títulos y seminarios}
\section{Otros formaciones}
%\cventry{2010}{Curso de Extensión: Introducción a las redes de datos e internet} {Universidad Nacional de Salta}{Salta Capital}{}{}
\cventry{Octubre 2011}{Curso de Extensión: Administración avanzada de GNU-Linux Debian} {Universidad Nacional de Salta}{Salta Capital}{}{}
\cventry{Noviembre 2013}{Curso de PostgreSQL: Administración y Desarrollo} {Escuela de Administración Publica}{Salta Capital}{}{}
%\cventry{2013 -- actualidad}{Especialización: Calidad Para el desarrollo de software} {INTI y Escuela de Administración Publica}{Salta Capital}{}{Serie de cursos introductorios y talleres orientados a la calidad de software, los temas desarrollados son Calidad de software, Normas ISO, CMMI, testing, patrones de diseño, arquitectura y metodologías ágiles: scrum y kanban}
%\cventry{Agosto 2007}{Cuokokrso de Introducción a Servidores }{Universidad de Sevilla}{Sevilla}{}{}
\cventry{Agosto 2013}{Curso de Introducción a la Calidad de Software}{EAP - INTI}{Salta Capital}{}{}
\cventry{Septiembre 2013}{Curso de Diseño de Software}{EAP - INTI}{Salta Capital}{}{}
\cventry{Octubre 2013}{Curso de Patrones de Diseño de Software}{EAP - INTI}{Salta Capital}{}{}
\cventry{Noviembre 2013}{Curso de Arquitectura de Software}{EAP - INTI}{Salta Capital}{}{}
\cventry{Diciembre 2013}{Curso de Normas ISO 9001:2008 Orientado a Empresas de Software}{EAP - INTI}{Salta Capital}{}{}
\cventry{Febrero 2014}{Curso de El testing como parte del proceso de Calidad de Software}{EAP - INTI}{Salta Capital}{}{}
\cventry{Marzo 2014}{Curso de Testing de aplicaciones Web}{EAP - INTI}{Salta Capital}{}{}
\cventry{Julio 2014}{Curso de Testing de Metodologías Ágiles}{EAP - INTI}{Salta Capital}{}{}
\section{Experiencia}
\cventry{Octubre 2012 - actualidad}{Analista programador}{Secretaria General de la Gobernación}{Salta Capital}{Análisis y desarrollo de sistemas informáticos para el apoyo de la gestión de distintos organismos del Gobierno Salteño}{}
\cventry{Julio 2016 - actualidad}{Programador}{Codedimension}{Salta Capital}{Desarrollo de frontends de sistemas varios para la promocion de Empresas o productos - codedimension.com.ar}{}
\cventry{Enero 2015 - Diciembre 2015}{Analista programador}{Clasapp.com}{Salta Capital}{Análisis y desarrollo de una aplicación web, esto incluye el desarrollo de un cliente web y un API RESTFull para acceder al sistema. El API debía servir tanto para el cliente web como para una aplicación Android}{}
%\section{Experiencia becada}
%\section{Experiencia Freelance}
%\section{Docencia}

%para ver el funcionamiento de subsecciones
%\subsection{Vocacional}

%\cventry{2005 - 2008}{Encargado de mantenimiento}{Ciber Argos}{Salta}{Era un negocio familiar en el que era encargado del mantenimiento de la red interna, computadoras e instalación eléctrica. El local fue cerrado}{}
%\cventry{Junio 2008 - Junio 2009}{Analista}{Empresa Y}{Sevilla}{}{}

\section{Idiomas}

%cvlanguage necesitan 3 llaves, cuestiones de formateo del curriculum
\cvlanguage{Castellano}{Nativo}{}
\cvlanguage{Inglés}{Nivel medio}{}
\closesection{}
%Salto de pagina
%\pagebreak{}

\section{Conocimientos técnicos}

%Los comandos cvcomputer necesitan 4 llaves
\cvcomputer{Programación}{PHP, Javascript, Java medio, Python medio}{}{}
%\cvcomputer{Científico}{LaTex}{}{}
\cvcomputer{Bases de Datos}{PostgreSQL, MySQL, SQLite}{}{}
\cvcomputer{Desarrollo web}{HTML5, CSS3}{}{}
\cvcomputer{Control de versiones}{Git avanzado, Subversion (SVN) basico}{}{}
\cvcomputer{Ofimática}{Open Office basico, Microsoft Office basico, LaTex}{}{}
\cvcomputer{Redes}{Armado y mantenimiento básico, configuración básica de routers}{}{}
%\cvcomputer{SSOO}{GNU/Linux (Debian, Ubuntu), Windows (XP, 7) instalación y mantenimiento en ambos}{}{}

\section{Otros datos de interés}
%Los cvitem necesitan dos llaves
\cvitem{Software Libre}{Soy usuario de software libre por lo que tengo conocimiento de las herramientas disponibles, aunque no por ello dudo a la hora de tener que usar otras soluciones disponibles cuando ello sea necesario. }
\cvitem{Personalidad}{Soy altamente adaptable a los diferentes entornos, siempre estoy predispuesto a mejorar y a aprender nuevas tecnologías, podría decirse que soy un autodidacta de nacimiento. Los mejores días de trabajo son esos en lo que se aprende algo nuevo.}
%\section{Intereses}
%\section{Más información}
\end{document}
